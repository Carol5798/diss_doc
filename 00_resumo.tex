\chapter*{Resumo}
A evolução das redes móveis, caracterizada pelo aumento da complexidade e pela dependência de soluções proprietárias, tem levado a desafios na integração de novas tecnologias.
Essas redes, que utilizam frequências de ondas milimétricas, são altamente suscetíveis a obstruções de Linha de Vista (LoS), o que leva a problemas de conectividade.
Abordar esses desafios requer soluções adaptativas, e a integração de arquiteturas de código aberto é essencial.

A Aliança O-RAN busca abordar esses desafios promovendo interfaces e arquiteturas abertas para aprimorar a interoperabilidade e a inovação nas Redes de Acesso Rádio (RANs).
Alavancar os princípios da O-RAN ao integrar Estações Base móveis (BSs) pode possibilitar uma melhoria na conectividade.
As BSs móveis oferecem uma abordagem de implantação dinâmica, prometendo atender a requisitos variados de Qualidade de Serviço (QoS) em diferentes contextos.

Nesse sentido, a convergência de sensoriamento e telecomunicações é esperada para revolucionar o desempenho da rede ao fornecer percepção do ambiente em tempo real.
A Visão Computacional (CV) tem um potencial significativo para aprimorar o desempenho da rede, proporcionando essa percepção para superar os desafios de propagação do sinal.
Ao processar e interpretar dados visuais, as redes podem identificar e mitigar proativamente obstáculos para garantir uma propagação de sinal ideal.
Essa capacidade é vital para manter uma força e qualidade de sinal consistentes em ambientes urbanos densos, onde os métodos tradicionais tem dificuldade ao lidar com obstáculos dinâmicos e móveis.

O principal objetivo da dissertação foi implementar uma RAN baseada em visão computacional, permitindo que uma BS perceba seu ambiente.
A principal contribuição foi a criação de um Módulo de Visão Computacional, responsável por extrair informações de vídeos e gerar mensagens relevantes de obstáculos para a RAN\@.
Para interpretar essas mensagens, foi desenvolvida uma xApp, integrando as mensagens com métricas da RAN, como a Relação Sinal-Ruído (SNR), visando aprimorar a percepção do ambiente do gNodeB (gNB).
A solução proposta foi validada experimentalmente considerando um caso de uso realista, demonstrando o potencial das RANs baseadas em visão na otimização do desempenho da rede.