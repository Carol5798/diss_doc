\chapter*{Resumo}
A evolução das redes móveis, caracterizada pelo aumento da complexidade e pela dependência de soluções proprietárias, leva a desafios na integração de novas tecnologias.
Redes que utilizam frequências de ondas milimétricas são particularmente vulneráveis a obstruções na Linha de Vista (LoS), resultando em problemas de conectividade.
Abordar esses desafios requer soluções adaptativas e a integração de arquiteturas de código aberto.
A Aliança Open Radio Access Network (O-RAN) busca enfrentar esses desafios, promovendo interfaces e arquiteturas abertas para melhorar a interoperabilidade e a inovação nas Redes de Acesso Rádio (RANs).
Utilizar os princípios da O-RAN na integração de Estações Base móveis (BSs) pode possibilitar conectividade de rede onipresente.
As BSs móveis oferecem uma abordagem de implementação dinâmica, prometendo atender a requisitos de Qualidade de Serviço (QoS) em diversos contextos.

Espera-se que a convergência de sensoriamento e telecomunicações revolucione o desempenho da rede, proporcionando percepção do ambiente em tempo real.
A Visão Computacional (CV) possui um potencial significativo para melhorar o desempenho da rede, oferecendo essa percepção para superar desafios de propagação de sinal.
Ao processar e interpretar dados visuais, as redes podem identificar e mitigar proativamente obstáculos para garantir a propagação ideal do sinal.
Essa capacidade é vital para manter a força e a qualidade do sinal consistentes em ambientes urbanos densos, onde métodos tradicionais enfrentam dificuldades com obstáculos dinâmicos e em movimento.

O principal objetivo desta dissertação foi implementar uma RAN auxiliada por visão computacional, permitindo que uma BS perceba seu ambiente.
A principal contribuição é um Módulo de Visão, responsável por extrair informações de vídeo e gerar mensagens relevantes de detecção de obstáculos para a RAN.
Para interpretar essas mensagens, foi desenvolvida uma xApp O-RAN, integrando-as com métricas da RAN, como a Relação Sinal-Ruído (SNR), aprimorando assim a percepção do ambiental do gNodeB (gNB).
A solução proposta foi validada experimentalmente em um caso de uso realista, demonstrando o potencial das RANs auxiliadas por visão na otimização do desempenho da rede.