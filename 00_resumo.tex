\chapter*{Resumo}
A evolução das redes móveis, caracterizada por uma crescente complexidade e dependência de soluções proprietárias, resulta em desafios na integração de novas tecnologias.
Redes sem fios que utilizam frequências de ondas milimétricas são particularmente vulneráveis à obstruçãod da Linha de Vista (LoS), o que leva a problemas de conectividade.
A resolução desses problemas requer soluções adaptativas e a integração de arquiteturas de código aberto.
A Aliança \textit{Open Radio Access Network} (O-RAN) procura endereçar esses desafios, promovendo interfaces e uma arquitetura abertas de referência para melhorar a interoperabilidade e a inovação nas Redes de Acesso Rádio (RANs).
A aplicação dos princípios da O-RAN na integração de Estações Base  (BSs) móveis abre caminho para conectividade de rede ubíqua.
As BSs móveis oferecem uma abordagem de implantação dinâmica, prometendo atender a diferentes requisitos de Qualidade de Serviço (QoS) em diversos contextos.

Espera-se que a convergência de entre visão e telecomunicações revolucione o desempenho das redes ao fornecer percepção do ambiente em tempo real.
A Visão Computacional (CV) possui potencial para melhorar o desempenho da redesuperando os desafios de propagação do sinal.
Ao processar e interpretar dados visuais, as redes podem identificar e mitigar proativamente obstáculos, garantindo uma propagação ótima do sinal.
Esta capacidade é vital para manter a potência do sinal recebido em ambientes urbanos densos, onde abordagens de implantação tradicionais enfrentam dificuldades com obstáculos dinâmicos e em movimento.

O principal objetivo desta dissertação foi implementar uma RAN auxiliada por visão, permitindo que uma BS percecione o ambiente circundante.
A principal contribuição é um Módulo de Visão, responsável por extrair informação de vídeo e gerar mensagens relevantes de detecção de obstáculos para a RAN\@.
Para interpretar essas mensagens, foi desenvolvida uma xApp O-RAN, correlacionando-as com métricas da RAN, como a Relação Sinal-Ruído (SNR), melhorando, assim a capacidade da BS percecionar o ambiente em tempo real.
A solução proposta foi validada experimentalmente num caso de uso realista, demonstrando o potencial das RANs auxiliadas por visão para a otimização do desempenho da rede.