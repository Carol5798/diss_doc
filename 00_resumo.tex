\chapter*{Resumo}
This document introduces the preliminary work on a perception-aided solution for 6G networks, addressing the increasing demands of users and overcoming limitations posed by proprietary solutions. It provides insights into the paradigms and applications of 5G and 6G networks and deployment approaches for Radio Access Networks (RANs). The document explores computer vision detection and tracking models, highlighting relevant work on blockage prediction, proactive handover, and O-RAN. It also presents the Horizon Europe CONVERGE research project. The main objective of this dissertation is to implement vision-based functionality into a 6G Base Station (gNB), paving the way for obstacle-aware networks. Specific objectives include implementing object detection and tracking mechanisms, developing a Vision-Enabled gNB Enhancement Application for real-time perception, creating a web service for information dissemination, and researching validation in networking scenarios. The document presents a proposed solution and outlines a work plan for the dissertation, detailing objectives that involve integrating computer vision tools with a service-oriented network architecture to support the operation of a perception-aided Base Station. The proposed solution aims to enable obstacle-aware networks by providing video-based information to a 6G Base Station repositioning. To validate the solution's effectiveness, a use case illustrating the synergy between computer vision, machine learning, and wireless communications will be considered, to demonstrate the potential of visual perception for seamless wireless connectivity.
