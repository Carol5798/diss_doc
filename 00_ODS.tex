
\chapter*{Sustainable Development Goals}
%\addcontentsline{toc}{chapter}{Abstract}
This dissertation takes into account the Sustainable Development Goals defined by the United Nations.
It directly addresses 4 out of 17 goals.

Those are:
\begin{enumerate}[label=]
    \item 8. Decent Work and Economic Growth
    \item 9. Industry, Innovation, and Infrastructure
    \item 11. Sustainable Cities and Communities
    \item 12. Responsible Consumption and Production
\end{enumerate}

In terms of decent work and economic growth, the convergence of sensing and telecommunications will drive technological innovation and create job opportunities.
Target 8.2 aims to achieve higher economic productivity through diversification, technological upgrading, and innovation.
This project aligns with Target 8.2 by leveraging CV to enhance the efficiency and capabilities of 5G networks, which is expected to boost economic productivity and generate high-tech job opportunities.

Emerging wireless networks, enhanced with computer vision and enabled by 5G, can significantly boost the reliability and performance of infrastructure, supporting sustainable development.
5G's high-speed and low-latency features are integral to Industry 4.0, allowing for advanced automation, predictive maintenance, and real-time monitoring.
This aligns with Target 9.1, focusing on developing quality, reliable, sustainable, and resilient infrastructure.
Additionally, by optimizing network performance and reducing energy consumption through CV techniques, this project contributes to Target 9.4, which aims to upgrade infrastructure and retrofit industries to be more sustainable.
Enhanced 5G networks can support smarter city management and reduce environmental impact by enabling efficient traffic management, pollution monitoring, and resource utilization through real-time data processing.
This contribution aligns with Target 11.6, which aims to reduce the adverse per capita environmental impact of cities.

The application of CV in 5G networks supports Industry 4.0 by optimizing supply chains and production processes, thereby reducing waste and promoting the efficient use of resources.
5G connectivity facilitates real-time data collection and analysis, essential for enhancing resource management and sustainability.
This aligns with Target 12.2, which focuses on achieving the sustainable management and efficient use of natural resources.

Emerging wireless networks, particularly through advancements in 5G technology, have the potential to significantly enhance economic growth, environmental protection, social equality, and urban development.
This dissertation supports economic development by driving innovation and enabling Industry 4.0 technologies, which boost productivity.
Optimized 5G networks can also contribute to environmental sustainability by improving resource efficiency and reducing emissions.
Enhanced connectivity and infrastructure provided by these networks facilitate equitable access to technology and services and promoting social equality.
Furthermore, smart city initiatives powered by advanced 5G technology improve urban living conditions by enabling more effective management of services and resources, ultimately enhancing overall quality of life.