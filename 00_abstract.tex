
\chapter*{Abstract}
%\addcontentsline{toc}{chapter}{Abstract}
The evolution of mobile networks, characterized by increasing complexity and reliance on proprietary solutions, leads to challenges in integrating new technologies.
Wireless networks using millimeter-wave frequencies are particularly vulnerable to Line-of-Sight (LoS) obstruction, resulting in connectivity issues.
Addressing these issues requires adaptive solutions and the integration of open-source architectures.
The Open Radio Access Network (O-RAN) Alliance seeks to address these challenges by proposing open interfaces and a reference, open architecture to promote interoperability and innovation in Radio Access Networks (RANs).
Leveraging O-RAN principles in integrating mobile Base Stations (BSs) paves the way for ubiquitous network connectivity.
Mobile BSs offer a dynamic deployment approach, promising to meet varying Quality of Service (QoS) requirements in diverse contexts.

The convergence of vision and telecommunications is expected to revolutionize network performance by providing real-time environmental awareness.
Computer Vision (CV) holds potential for enhancing network performance by overcoming signal propagation challenges.
By processing and interpreting visual data, networks can proactively identify and mitigate obstacles to ensure optimal signal propagation.
This capability is vital for maintaining received signal power in dense urban environments, where traditional deployment approaches struggle with dynamic and moving obstacles.

The main objective of this dissertation was to implement a vision-aided RAN, enabling a BS to perceive the surrounding environment.
The main contribution is a Vision Module, responsible for extracting information from video and generating relevant obstacle detection messages to the RAN\@.
In order to interpret these messages, an O-RAN xApp was developed, correlating them with RAN metrics, such as Signal to Noise Ratio (SNR), improving the BSs' capacity to perceive the environment in real-time.
The proposed solution was experimentally validated with a realistic use case, showcasing the potential of vision-aided RANs in optimizing network performance.