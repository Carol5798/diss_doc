
\chapter*{Abstract}
%\addcontentsline{toc}{chapter}{Abstract}
The evolution of mobile networks, characterized by increasing complexity and reliance on proprietary solutions, leads to challenges in integrating new technologies.
Networks, using millimeter-wave frequencies, are particularly vulnerable to Line-of-Sight (LoS) obstruction, resulting in connectivity issues.
Addressing these challenges requires adaptive solutions and the integration of open-source architectures.
The Open Radio Access Network (O-RAN) Alliance seeks to address these challenges, by promoting open interfaces and architectures to enhance interoperability and innovation in Radio Access Networks (RANs).
Leveraging O-RAN principles in integrating mobile Base Stations (BSs) can enable ubiquitous network connectivity.
Mobile BSs offer a dynamic deployment approach, promising to meet varying Quality of Service (QoS) requirements in diverse contexts.

The convergence of sensing and telecommunications is expected to revolutionize network performance by providing real-time environmental awareness.
Computer Vision (CV) holds significant potential for enhancing network performance by offering real-time environmental awareness to overcome signal propagation challenges.
By processing and interpreting visual data, networks can proactively identify and mitigate obstacles to ensure optimal signal propagation.
This capability is vital for maintaining consistent signal strength and quality in dense urban environments, where traditional methods struggle with dynamic and moving obstacles.

The main objective of this dissertation was to implement a vision-aided RAN, enabling a BS to perceive its environment.
The main contribution is a Vision Module, responsible for extracting information from video and generating relevant obstacle detection messages to the RAN\@.
In order to interpret these messages, an O-RAN xApp was developed, integrating them with RAN metrics, such as Signal to Noise Ratio (SNR), thereby enhancing the environmental perception of the gNodeB(gNB).
The proposed solution was experimentally validated in a realistic use case, showcasing the potential of vision-aided RANs in optimizing network performance.