\chapter{Conclusion}\label{ch:conclusion}

\section{Conclusions}\label{sec:conclusions}
The goal of this dissertation was to implement a vision-aided RAN\@.
The proposed solution enabled the use of use Computer Vision (CV) techniques to extract relevant information and introduce vision-based messages within the RAN\@.
Leveraging the O-RAN architecture, this dissertation demonstrates the integration of sensing and telecommunications technologies.
The convergence of these fields motivates the potential of CV to enhance the efficiency and intelligence of 5G and 6G networks.

On one hand, the O-RAN Alliance develops and provides open and interfaces.
The O-RAN architecture leverages SDN and NFV principles, representing a shift from closed proprietary systems.
This allows interfaces, such as the E2 interface, to extract information from the RAN, with the development of software applications, such as a xApp.
On the other hand, solutions such as vision-aided BSs' allow 5G networks to offer better QoS and QoE to clients.
The introduction of vision in the RAN architecture enables environmental perception, in order to tackle LoS blockages.
This enables the gNBs to make more informed decisions regarding their placement, especially in dynamic scenarios.

The proposed solution leverages CV techniques and O-RAN architecture, to implement a standalone 5G network, with a vision-aided RAN\@.
The implementation utilized the open-source software of OpenAirInterface for the 5G Core Network and RAN\@.
The Near-RT RIC was implemented using the FlexRIC software.
In order to incorporate CV into our system, a novel Vision Module solution was developed using OpenCV, Ultralytics YOLO, and BoT-SORT for object detection and tracking.
USRP B210 SDRs were used to enable communications between a UE and a gNB, enabling the creation of a proof-of-concept.
Finally, the developed xApp monitored and processed the data, demonstrating the practical application of CV in enhancing 5G network performance.

The proposed Vision Module is able to extract data captured from a real-time video feed and generate relevant information for the RAN\@.
The xApp extracts SNR measures, with a periodicity close to the video processing rate, in order to combine them.
The xApp combines such information and triggers recommendations for repositioning the gNB, in order to maintain LoS with a UE\@.

Throughout the development of the proposed solution, we encountered several implementation challenges.
One significant issue was the lack of documentation for different software packages.
When it comes to OpenAirInterface and FlexRIC, this difficulty hindered the deployment of the 5G network.
As they are complex software packages lacking detailed information,this made it difficult to understand setup and configuration requirements, leading to numerous trial-and-error situations.
Software instability, version control issues, frequent bugs, and compatibility problems with new updates further complicated the deployment of the OAI\@.
Additionally, there was limited assistance from the development team.

Similarly, the \emph{asn1c} library also suffered from insufficient documentation and code maintenance, further complicating our implementation efforts and the integration of messages with FlexRIC\@.

When it came to hardware requirements, both OAI and the tracking solution were resource intensive.
This led to occasional crashes and errors.
For tracking, this required a lot of experimenting with different tracking solutions to improve the processing rate of the video frames.
Despite FlexRIC being less resource intensive, occasional crashes occurred when developing the xApp using an emulated gNB\@.

Additionally, we faced instabilities caused by the SDRs, disrupting the progress of our tests.
These instabilities often required us to relaunch the software or physically attach and detach the board to restore functionality, consuming a considerable amount of time.
They were mainly related to sending or receiving samples to and from the SDRs.

These difficulties made it arduous to test the system as a whole, but they were overcome.
We managed to test and validate each component individually and jointly, and deploying a testbed to validate our system in representative scenarios.

The objectives of this dissertation were fully met, resulting in three contributions: 1) the introduction of video-based information into a 5G network based on O-RAN architecture; 2) the development of an xApp that joins video information and RAN metrics, enhancing the gNB's capabilities with environmental perception; and 3) the validation and evaluation of the proposed solution's performance in reference networking scenarios, including a proof-of-concept for vision-aided networking solutions.
Our solution serves as a baseline for integrating CV into the RAN\@.

\section{Known Limitations and Future Work}\label{sec:fut_work}
While this dissertation successfully met its objectives, the solution has some limitations, which present opportunities for enhancements and future work.

The VM may benefit from several improvements, including the optimization of parameters, such as the capturing frame rate and the persistence time, and determining the optimal number of frames required for the prediction capabilities of the messages, especially the Prior Blockage.
Improving the accuracy of generating Prior Blockage messages may also benefit from using the relative velocity between the ArUco marker and the obstacle.
The current solution only uses the marker's current location to project the placement of the obstacle.
Additionally, employing a 3D mapping solution to obtain depth information would enable the extraction of more precise locations of obstacles and UEs.
Scaling the solution to manage multiple UEs simultaneously is another area for improvement.
The current implementation also faced performance constraints due to hardware limitations.
Testing the solution on dedicated hardware would improve the real-time operation and response of the system.

Utilizing the E2 interface for information exchange between the RAN and the Vision Module may enhance communications efficiency.

Future work may also involve integrating the xApp with a robotic platform, enabling it to autonomously reposition the gNB\@.
Additionally, the performance limitations of the SDRs need to be addressed to ensure more robust and reliable communications.

Finally, extending the experimental tests to include mmWave carrier frequencies may provide valuable conclusions about the system's performance in different operational scenarios, further validating and evaluating approaches to improve the state of the art.



% refine parameters in the vision module -
% periodicity of messages
% future blockage enhancement such as no duplicates

% test in better hardware?

%\begin{enumerate}
%    \item Only predicts for one UE
%    \item Camera resolution needs to be at least 360p -- check
%    \item Camera FPS
%    \item Computational power limited. Could use other tracking solutions
%    \item Lighting condition
    
%\end{enumerate}


%FUTURE WORK - have 2 cameras in order to have depth information and extract (x,y,z) location of the obstacles and UEs. Scale the work for several UEs. Use interface E2 to communicate with the xApps.