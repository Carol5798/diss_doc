\chapter{Conclusion}\label{ch:conclusion}

\section{Conclusions}\label{sec:conclusions}
The goal of this dissertation was to implement vision-based RAN\@.
The proposed solution allowed us to use Computer Vision techniques to extract relevant information and introduce messages to the RAN\@.

Leveraging the O-RAN architecture, this work demonstrated the seamless integration of sensing and telecommunications technologies.
The convergence of these fields highlights the potential of CV to significantly improve the efficiency and intelligence of 5G networks.

The implementation utilized the open-source software of OpenAirInterface for the 5G Core Network and RAN\@.
The Near-RT RIC was implemented using FlexRIC software.
The Vision Module was developed using OpenCV, Ultralytics YOLO, and BoT-SORT for object detection and tracking.
USRP B210 SDR boards were used to enable communications between UE and gNB\@.
The developed xApp monitored and processed the data, showcasing the practical application of CV in enhancing 5G network performance.

Through the development of the proposed solution we faced some challenges.

In summary, this dissertation illustrates the feasibility and advantages of integrating CV technologies within 5G networks, paving the way for future advancements in telecommunications.



The open-source software used to implement the 5G Core Network and RAN was OpenAirInterface.
The developed VM used OpenCV, Ultralytics YOLO and BoT-SORT to detect and track objects.
USRP B210 SDR boards were used to enable communications between UE and gNB\@.
Finally, the developed xApp monitored.

\section{Known Limitations and Future Work}\label{sec:fut_work}
% refine parameters in the vision module - optimal number of frames to obtain tracks and to prediction
% periodicity of messages
% future blockage enhancement such as no duplicates
% test in mmWaves
% test in better hardware?

%\begin{enumerate}
%    \item Only predicts for one UE
%    \item Camera resolution needs to be at least 360p -- check
%    \item Camera FPS
%    \item Computational power limited. Could use other tracking solutions
%    \item Lighting condition
    
%\end{enumerate}


%FUTURE WORK - have 2 cameras in order to have depth information and extract (x,y,z) location of the obstacles and UEs. Scale the work for several UEs. Use interface E2 to communicate with the xApps.