\chapter{System Validation}\label{ch:validation}

This chapter presents the validation of the proposed solution.
The validation process is broken down into several key sections, each addressing different components of the system.
Section~\ref{sec:meth} outlines the methodology used for validation.
Section~\ref{sec:core_network} describes the core network setup. Section \ref{sec:flexric} details the FleXRIC framework used in the system.
Section \ref{sec:gnb} covers the gNB (gNodeB) configuration and validation. Section \ref{sec:ue} addresses the User Equipment (UE) setup and tests.
Section \ref{sec:cv_module} discusses the computer vision module and its integration. Section \ref{sec:mm_xapp} focuses on the Mobility Management xApp and its role in the system .
Section \ref{sec:use_case} presents a specific use case to demonstrate the system's functionality.
Finally, Section \ref{sec:discuss} provides a discussion of the results and insights gained from the validation process.

\section{Methodology}\label{sec:meth}
This section describes the overall methodology employed to validate the system. It includes details on the experimental setup, data collection methods, and the criteria used for evaluation.

\section{Core Network}\label{sec:core_network}
In this section, the core network configuration and validation are discussed. It includes the setup of network elements, their interactions, and performance metrics.
In order to make sure that all Core Network components are working properly, we performed a test.
The deployment of the Core Network is done through Docker containers. Figure \ref{} presents the command for running the Core Network setup script.

% insert figure with command to up the core .
After the initialization of the Docker containers, it is possible to see the logs of the setup script. They test if the containers are healty.

Then, we verified that all the containers had connectivity using the interface created in the Host OS, using the ping tool.
Each interface sends requests to each network component according to the table \ref{}.

% insert logs of the core initialization

% insert ping for all core components

Successfully concluding this tests ensures that the 5G Core Network is fully operational.
Wireshark logs can also be inspected to analyze the traffic in the core inteface.

% include wireshark log of some traffic in demo-oai


\section{FleXRIC}\label{sec:flexric}
This section provides an overview of the FleXRIC framework used in the system. It explains how FleXRIC is implemented and validated within the network environment.

\section{gNB}\label{sec:gnb}
Here, the focus is on the gNB (gNodeB) setup and validation. It covers the hardware and software configuration, connectivity tests, and performance evaluation.

\section{UE}\label{sec:ue}
This section discusses the User Equipment (UE) configuration and validation. It includes the setup of devices, connection procedures, and performance assessments.

\section{Computer Vision Module}\label{sec:cv_module}

This section discussed the computer vision module, explaining its integration into the system and the validation of its functionalities, such as detecting objects and sending messages to the xApp. The computer vision module is the main developed component in the system of this dissertation, enabling real-time object detection and tracking to enhance dynamic network management.

The computer vision module serve as a server to the Mobility Management xApp. Their interface ensures reliable data exchange using a socket connection, leveraging an ASN.1 file to standardize the structure of the messages. 

To ensure the correctness and reliability of the computer vision module, a series of validation tests were conducted. These tests were designed to evaluate both the processing capabilities of the module and the accuracy of the message exchange between the vision module and the xApp.

The processing time of each frame was measured to assess the real-time performance of the computer vision module. This was critical to ensure that the module could keep up with dynamic environments, such as an office setting where people and objects are constantly moving. The tests showed that the processing times were within acceptable limits, allowing for timely detection and response.
% show some math on why this result is sufficient
% plot some results

A reference video was used to evaluate the detection and tracking results of the computer vision module. This video, containing typical office movements like people walking and objects being moved, was processed to check for detection accuracy and tracking consistency. The results confirmed that the module could accurately detect and track objects, validating its effectiveness in a real-world scenario.

% image of a frame of the video and the corresponding messages

Print statements were used on the server side to verify the correct formatting, coding, and decoding of the messages. This step was crucial to ensure that the messages sent from the computer vision module to the xApp were correctly structured and could be properly interpreted upon receipt.

On the client side, print statements were employed to confirm the correct reception of the messages. This validation step ensured that the messages transmitted through the socket connection were received intact and could be  correctly processed by the xApp.

To further validate the communication, Wireshark was used to capture SCTP packets containing the messages exchanged between the server and client. This capture provided a detailed view of the message flow, confirming that the messages were being transmitted as expected without any loss or corruption.

% image of the wireshark capture

% MEASURE IF THE SCTP MESSAGES ARE BOTTLENEKING THE APPLICATION.

% maybe define a periodicity for it.


The performance metrics achieved by the computer vision module were deemed sufficient for the intended application. In scenarios such as office environments, where there are frequent but predictable movements (e.g., people walking, objects being moved), the module demonstrated robust performance. The real-time processing capability ensured that the system could react promptly to changes, maintaining effective communication and mobility management.



\section{Mobility Management xApp}\label{sec:mm_xapp}
This section focuses on the Mobility Management xApp and its role in the system. It details the design and implementation of the xApp, how it interfaces with other components, and the results of its validation.

\section{Use case}\label{sec:use_case}
In order to validate the implemented solution, a use case testing scenario was established. In an indoor environment, the system followed the architecture presented in Figure \ref{fig:}. 


The goal of test was to access the functionality of the whole system, considering maintaining end-to-end connection between the UE and the external DN. The Mobile RAN positioning is defined by the mobility management xApp, based on data collected from the Computer Vision Module and the radio metrics collected from the RAN via the Near-RT RIC. It aimed at maintaining the channel quality, or increasing it whenever possible.

The use case shows the system's capabilities in two test scenarios, described in the following subsections. 

\subsection{Scenario 1: UE Blocked by an Obstacle}


In the first scenario, the User Equipment (UE) encounters an obstacle that blocks its path.
The system identifies the obstacle and predicts the time expected for the blockage.
The robotic platform, equipped with the computer vision module and the Mobility Management xApp, detects this impending blockage.
As a result, the platform can proactively move to avoid the obstacle, ensuring uninterrupted communication for the UE.

\subsection{Scenario 2: UE Moving Away from the gNB}

The second scenario involves the UE moving further away from the gNB (gNodeB).
In this case, if no obstacle is identified and the Signal-to-Noise Ratio (SNR) lowers, the system assumes that the UE has moved away from the gNB.
The robotic platform, utilizing the Mobility Management xApp, responds by moving towards the UE to maintain optimal communication quality.
This dynamic adjustment ensures that the UE remains within the effective communication range of the gNB.


\section{Discussion}\label{sec:discuss}
This section provides a discussion on the results obtained from the validation process.
It includes insights, lessons learned, and potential areas for improvement in future iterations of the system.
